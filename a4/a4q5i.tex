First, consider the coordinator:
when the coordinator reboots, the coordinator should recover a list of votes from servers that responded successfully.
Based on the logs, if at least one server responded with a NO, then the coordinator sends \emph{global abort} messages to the participants; recovery is complete and 2PC goes on as usual.
Otherwise, the coordinator resends \emph{prepare} messages to participants that are still missing votes; at this stage, recovery is complete.

On the other hand, participants don't need to recovery anything and are waiting for messages that come from the coordinator.
There are 3 cases that participants wait for:
\begin{itemize}
\item
The reply has not reached the coordinator.
In this case, these participants receive a \emph{prepare} message, and they resend a YES/NO vote.
Since the coordinator may fail repeatedly, these participants may wait and resend votes indefinitely until it is in the second case.
\item
The reply reaches the coordinator.
In this case, participants do not receive any message until the election is complete.
When the election is complete, all participants receives a \emph{global commit/abort} message.
At this point, recovery is complete.
\item
The reply has not reached the coordinator, but some participants who voted NO do.
Participants realize this if the coordinator sends a \emph{global abort} message.
At this point, recovery is complete.
\end{itemize}
