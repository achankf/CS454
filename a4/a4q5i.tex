First, consider the coordinator:
when the coordinator reboots, the coordinator should recover from the log for the following items:
\begin{itemize}
\item
the fact that 2PC has begun --- this fact tells the coordinator from which point to continue
\item
a list of servers who responded with a vote --- the description implies that the coordinator could get some replies before it crashed, so some votes may have gotten into the log
\end{itemize}
Based on the logs, if at least one servers responded with NO, then log a \emph{global abort} entry and then send \emph{global abort} messages to the participants; recovery is complete and 2PC goes on as usual.
Otherwise, resend \emph{prepare} messages to servers that the coordinator has not yet seen a reply; at this stage, recovery is complete.

On the other hand, participants don't need to recovery anything and are waiting for messages that come from the coordinator.
Waiting is ok because participants cannot continue without a coordinator anyways.
There are 3 cases that participants wait for:
\begin{itemize}
\item
The reply has not reached the coordinator.
In this case, these participants receive a \emph{prepare} message, and they resend a YES/NO vote.
Since the coordinator may fail repeatedly, these participants may wait and resend votes indefinitely until it is in the second case.
\item
The reply reaches the coordinator.
In this case, participants do not receive any message until the election is complete.
When the election is complete, all participants receives a \emph{global commit/abort} message, and is added to the logs.
At this point, recovery is complete.
\item
The reply has not reached the coordinator, but some participants who voted NO do.
Participants realize this if the coordinator sends a \emph{global abort} message.
Of course, the message is logged.
At this point, recovery is complete.
\end{itemize}
