For the coordinator, it should wait for one of the following two messages:
\begin{itemize}
\item
\emph{prepared}:
in this case, the coordinator knows that a participant has failed and has rebooted.
Thus, the coordinator resends \emph{global commit/abort} and wait again.
To avoid an infinite failure loop, the coordinator should keep track of the number of times a participant resend this message.
I think the coordinator should force a global abort if the number reaches a limit (say, 10 times), because the participant probably has a big problem that require operators' intervention.
Depending on the replication algorithm, I think the coordinator should tell that participant to shutdown until the issues become fixed, and new requests will be run without the participant.
\item
If the coordinator receives a notification that a participant has completed the \emph{global commit/abort}, then the coordinator can proceed to the clean up stage.
Recovery is complete.
\end{itemize}

For the participants:
if the participant has failed many times in a row (i.e.\ needs operators' intervention), then it should be told by the coordinator to shutdown.
Otherwise, it must read its logs and rewind to the previous 2PC states.
That is, the participant should resend the YES/NO vote to the coordinator.
Notice that the \emph{prepared} message must the same as the previous one before the participant failed, because the assignment description implies that the vote decision is recorded in the logs.
Then, the participant should wait for \emph{global commit/abort} and the recovery is complete.
