For the coordinator, it should wait for one of the following two messages from the failed participants:
\begin{itemize}
\item
\emph{prepared}:
in this case, the coordinator knows that a participant has failed and has rebooted.
Thus, the coordinator resends \emph{global commit/abort} and wait again.
To avoid infinite failure loops, the coordinator should keep track of the number of times a participant resend this message, and the coordinator should send a global abort if a limit is reach (say, 10 times).
Depending on whether the participant is a backup replica or not, I may consider to shut it down.
\item
If the coordinator receives a notification that a participant has completed the \emph{global commit/abort}, then the coordinator can proceed to the clean up stage.
Recovery is complete.
\end{itemize}

For the participants, it must reboot to the last good 2PC states.
That is, the participant should resend the YES/NO vote to the coordinator.
Notice that the \emph{prepared} message must be the same as the previous one that was sent before failure, because the decision must be recorded in the logs before the failure.
At this stage, recovery is complete and the participant should be waiting for a \emph{global commit/abort} message from the coordinator.
