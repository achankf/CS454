\section{Error Numbers}
The following errors/warnings are defined in an enum in {\tt common.hpp}.
\begin{itemize}
\item
{\tt EXECUTE\_WITHOUT\_REGISTER} (2): this is a warning that the server runs {\tt rpcExecute} without registering any functions.
In case this happens, {\tt rpcExecute} immediately terminates and return this warning.
\item
{\tt SKELETON\_UPDATED} (1): this warnings that the server re-registers a method, and only the skeleton is updated.
\item
{\tt OK} (0): this means successful.
\item
{\tt BAD\_FD} (-1): this happens when {\tt Sockets} couldn't create a socket.
\item
{\tt BINDER\_UNAVAILABLE} (-2): the binder is down.
\item
{\tt CANNOT\_ACCEPT\_CONNECTION} (-3): this happens when {\tt Sockets} cannot accept a connection.
\item
{\tt CANNOT\_BIND\_PORT} (-4): this happens when {\tt Sockets} cannot bind a port.
\item
{\tt CANNOT\_CONNECT\_TO\_SERVER} (-5): this happens when a client cannot connect to a server.
\item
{\tt CANNOT\_LISTEN\_PORT} (-6): the binder/server cannot set a port to listen to incoming messages.
\item
{\tt CANNOT\_RESOLVE\_HOSTNAME} (-7): the local name directory (binder/clients/servers) cannot find the hostname.
\item
{\tt CANNOT\_START\_CONNECTION} (-8): binder/clients/servers cannot start a connection.
\item
{\tt CANNOT\_WRITE\_TO\_SOCKET} (-9): cannot write to a socket.
\item
{\tt FUNCTION\_ARGTYPES\_ARE\_INVALID} (-10): {\tt argTypes} is {\tt NULL}
\item
{\tt FUNCTION\_NAME\_IS\_INVALID} (-11): functions whose name is null, or {\bf not} $0 < $ length $\le 63$ (this excludes the null terminator).
Notice valid names must have a length that is strictly greater than 0.
\item
{\tt FUNCTION\_NOT\_REGISTERED} (-12): client tries to call a function that a server did not register.
\item
{\tt HAS\_ALREADY\_INIT\_SERVER} (-13): the server calls {\tt rpcInit} more than once.
This error number is returned by {\tt rpcInit}.
\item
{\tt HAS\_RUN\_EXECUTE} (-14): the server attempts to run {\tt rpcExecute} more than once.
\item
{\tt NOTHING\_TO\_RECEIVE} (-15): {\tt select()} tells {\tt Sockets} to fill up the read buffer, but there's nothing to fill.
\item
{\tt NOTHING\_TO\_SEND} (-16): {\tt select()} tells {\tt Sockets} to clear the write buffer, but there's nothing to write.
\item
{\tt NOT\_A\_CLIENT} (-17): a server (i.e. {\tt rpcInit()} has run) tries to run client methods: {\tt rpcCall}, {\tt rpcCacheCall}, and {\tt rpcTerminate}
\item
{\tt NOT\_A\_SERVER} (-18): a client tries to run server methods: {\tt rpcInit}, {\tt rpcRegister}, and {\tt rpcExecute}.
\item
{\tt NO\_SERVER\_AVAILABLE} (-19): the name directory cannot find a server to complete requests.
\item
{\tt REMOTE\_DISCONNECTED} (-20): the remote machine disconnected before the calling machine gets a reply.
\item
{\tt SKELETON\_FAILURE} (-21): the skeleton function returns a negative value.
\item
{\tt SKELETON\_IS\_NULL} (-22): the provided skeleton is a {\tt NULL} pointer.
\item
{\tt SERVER\_HAS\_NO\_AVAIL\_THREADS} (-23): the server rejects a request because it doesn't have any free worker threads.
This only happens when you call {\tt rpcCacheCall}, because it defies round-robin.
\item
{\tt TERMINATING} (-24): this represents the server is terminating -- this is probably not a ``public-facing" errno.
\item
{\tt UNREACHABLE} (-100): unreachable codes reached; in other words, gg.
\end{itemize}
