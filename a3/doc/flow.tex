\section{Program Flow}
I will try to describe the flow of the system.

\subsection{Client}
Ignoring {\tt rpcCacheCall}, the system goes as follows.
\begin{itemize}
\item
The client initialize {\tt Global} variables.
\item
The client calls {\tt rpcCall}.
\item
{\tt rpcCall} ask the binder for a server suggestion
\item
If the binder finds a good suggestion, then reply the client with the server's id.
Otherwise, send a failure.
In either cases, the binder attaches partial logs of the name directory based on client's version in the request header.
\item
If the client doesn't get a valid suggestion, then end {\tt rpcCall}.
Otherwise, the client send an execute request to the server.
\item
The server gets the request, run it, and then reply to the client.
The server also send partial logs to the client, in case other servers join and caused this server is forced to update its name directory.
\item
The client applies to logs, get the result, and then end {\tt rpcCall}.
\item
It is possible that later the client calls for a terminate request to the binder.
\end{itemize}

\subsection{Server}
Generally, the flow goes as follows.
\begin{itemize}
\item
The server initialize {\tt Global} variables and call {\tt rpcInit} separately.
\item
During {\tt rpcInit}, the server sends a {\tt I\_AM\_SERVER} request to the binder and wait for its id in a {\tt SERVER\_OK} reply.
\item
Then the server register functions.
\item
Then the server run {\tt rpcExecute}.
At this point, the server sends a {\tt NEW\_SERVER\_EXECUTE} to the binder, and the binder will broadcast this request to all servers, forcing them to send a {\tt ASK\_NS\_UPDATE}, which updates all servers' name directory.
Notice that it is possible for the binder to catch some dead servers, which causes the binder to remove them from its name directory, effectively incrementing the versoin.
Thus, it is a possibility that {\bf not every server will be synchronized to the same, latest version}.
However, we will see in later sections that this is ok.
\item
Meanwhile, the server may get a terminate request from {\bf anyone}.
Thus, it will ask the binder for confirmation through a {\tt CONFIRM\_TERMINATE} request.
If the binder agrees the confirmation, then the server will call {\tt Tasks::terminate()}, which is a blocking call that {\bf may not necessarily return, if a thread is running a skeleton function that has an infinite loop} -- I assume that all skeleton functions will eventually halt.
Once all tasks are ended, the server will exit gracefully.
Note that the binder may terminate even before the server does so, assuming the server has long-running tasks that {\bf will eventually terminate}.
\end{itemize}

\subsection{The Binder}
Unlikely clients/servers, the requests can arrive to the binder in any order, but the binder handles them in a gigantic switch statement in a sequential manner.
In particular, the binder handles the following requests:
\begin{itemize}
\item
{\tt I\_AM\_SERVER}
\item
{\tt REGISTER}
\item
{\tt LOC\_REQUEST}
\item
{\tt NEW\_SERVER\_EXECUTE}
\item
{\tt CONFIRM\_TERMINATE}
\item
{\tt ASK\_NS\_UPDATE}
\end{itemize}
